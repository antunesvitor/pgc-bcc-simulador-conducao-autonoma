% ---
% Inicia os apêndices
% ---
\begin{apendicesenv}

% Imprime uma página indicando o início dos apêndices
\partapendices

% ----------------------------------------------------------
\chapter{Sobre os testes feitos e como reproduzi-los}
% ----------------------------------------------------------

Os testes dos quais a seção de resultados se refere estão disponíveis em vídeo nesta URL do Google Drive: \href{https://drive.google.com/drive/folders/1WcdHMeFQLfDubvYvLJnqbobnysT5pJoq?usp=sharing}{https://drive.google.com/drive/folders/1WcdHMeFQLfDubvYvLJ\\nqbobnysT5pJoq?usp=sharing}.

Como reproduzir os testes está descrito na página do repositório oficial do projeto (\href{https://github.com/antunesvitor/SimuladorDeConducao}{https://github.com/antunesvitor/SimuladorDeConducao}), porém de modo a deixar este texto completo com toda a informação necessária as instruções serão disponíveis aqui também.

\section*{Requisitos}

\begin{itemize}
    \item Sistema Operacional: Windows 11¹  ou Linux²
    \item Unity Hub
    \item Unity3D editor versão 2022.3.13f1³
    \item Git (opcional)
\end{itemize}

¹ A Unity3D está disponível para Windows 10 também, mas este projeto não foi testado nele, provavelmente é possível reproduzir/treinar sem problema algum, no Win10.

² As versões de Linux suportadas oficialmente é O Ubuntu 16.04, 18.04 e CentOS 7.  O desenvolvimento do projeto foi feito no Arch Linux que NÃO é oficialmente suportado.

³ Pode ser possível executar em outras versões 2022.3.x.

\section*{Como reproduzir} 
Aqui é explicado como reproduzir os testes feitos no projeto. Eles não obterão o exato resultado mostrado no projeto dado a natureza estocástica do agente e excesso de possíveis estados, porém um desempenho muito próximo é esperado.

\subsection*{Instalando e abrindo o projeto}

 \begin{enumerate}
    \item Primeiro baixe e instale o Unity Hub.
    \begin{itemize}
        \item Windows: baixe e instale pelo site oficial da Unity (\href{https://unity.com/download}{https://unity.com/download}).
        \item Arch Linux: instale o pacote na AUR (\href{https://aur.archlinux.org/packages/unityhub}{https://aur.archlinux.org/packages/unity\\hub}). Para isso terá que usar um gerenciador de pacotes da AUR como o yay (\href{https://github.com/Jguer/yay}{https://github.com/Jguer/yay}) ou o paru (\href{https://github.com/morganamilo/paru}{https://github.com/morganamilo/\\paru}).
        \item Para as distribuições Linux oficialmente suportadas siga as instruções na documentação (\href{https://docs.unity3d.com/2020.1/Documentation/Manual/GettingStartedInstallingHub.html}{https://docs.unity3d.com/2020.1/Documentation/Manual/GettingS\\tartedInstallingHub.html}.
    \end{itemize}
    \item Após isso clone ou baixe o repositório do projeto (\href{https://github.com/antunesvitor/SimuladorDeConducao}{https://github.com/antunesvitor/Si\\muladorDeConducao}).
    \item Abra o Unity Hub, vá em ``Open'' e navegue até o diretório onde você baixou o projeto e clique abrir. É possível que apareça um aviso de que a versão do editor não está instalada no seu computador, para isso existem duas opções:
    \begin{itemize}
        \item (Recomendado) Baixar a versão exata do projeto: vá à página de arquivo de editores da Unity (\href{https://unity.com/releases/editor/archive}{https://unity.com/releases/editor/archive}) clique na aba "Unity 2022.X"{} e clique no botão  "Unity Hub"{} na versão 2022.3.7;
        \item (Não recomendado) Alterar a versão do projeto: quando o aviso aparecer clique em "Choose another editor version"{} e então em "Install Editor Version"{} ele te apresentará as versões LTS disponíveis para download;
    \end{itemize}
    \item Após isso basta clicar duas vezes para abrir o projeto.
 \end{enumerate}

\subsection*{Testando os modelos}
Para testar os modelos apropriadamente, é preciso testar um por vez e ajustar o projeto de acordo, alguns dos "cérebros"{} (arquivos .onnx na pasta Brains são a política definida pelo treino, a rede neural) são de trajetos específicos.

\begin{enumerate}
    \item Selecione o objeto Car na hierarquia do projeto.
    \item Arraste o "cérebro"{} que deseja treinar para o campo \textit{Model} no inspetor à esquerda do editor.
    \item Desabilite todos os \textit{paths} que não for usar (consultar \nameref{nomenclatura}).
    \item Agora basta apertar o botão \textit{play} no topo ao centro e o veículo já estará sendo conduzido pelo cérebro. (Obs. garanta que os campos \textit{Behaviour Type} e \textit{Inference Device} estejam ambos em \textit{Default}  ou  \textit{Inference Only} e \textit{CPU} respectivamente, caso contrário não funcionará).
\end{enumerate}

\subsection*{Nomenclatura dos cérebros}\label{nomenclatura}
Os "cérebros"{} estão na pasta Assets/Brains e seguem o seguinte padrão de nome:

CarSim<algoritmo><versão>\_<path>-<id>

<algoritmo> serve para identificar com qual algoritmo foi treinado. Possíveis valores são: PPO, SAC, PPO+BC, PPO+GAIL, PPO+GAIL+BC, etc. 

<versão> é um número que identifica a versão do veículo, durante o projeto diversas versões foram criadas, atualmente só há uma versão dele (3.1), porém esta versão foi posta para deixar claro que possa haver mais de um agente em versões diferentes no mesmo ambiente.

<path> a rota a qual aquele cérebro treinou, se habilitar uma rota diferente do cérebro ele provavelmente não irá desempenhar bem a tarefa. O nome é igual aos GameObjects localizados abaixo do GameObject SpawnPointManager. O "GERAL"{} significa que ele treinou em todas as rotas.

<id> opcional, um identificador a mais para diferenciar um cérebro de outro caso os valores acima sejam iguais.


% ----------------------------------------------------------
% \chapter{Segundo apêndice com título tão grande quanto se queira porque ele já faz a quebra de linha da coisa toda}
% % ----------------------------------------------------------
% \lipsum[51-53] % Texto qualquer. REMOVER!!

\end{apendicesenv}
% ---
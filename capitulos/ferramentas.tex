\chapter{Materiais e Métodos}\label{cap:ferramentas}

Para criar o projeto foi usado a Unity3D na versão \textbf{2022.3.7f1}. Acerca do \textbf{ML Agents}, há dois pacotes, o primeiro para o editor da Untiy3D que adiciona os componentes e classes de \textbf{RL} \textbf{IL}, esta está na versão \textbf{2.0.1}. O outro pacote seria o pacote Python, instalado via \textbf{pip}(gerenciador de pacotes do Python), este encontra-se na versão \textbf{0.30.0}. 

Como foi dito, para fazer o agente treinar é necessário criar um ambiente Python que interage com o editor da Unity informação, para isto foi usado Python na versão \textbf{3.9.13}, a principal dependência Python do \textbf{mlagents} é o \textit{framework} \textbf{PyTorch}, neste projeto usa-se a versão \textbf{1.7.1}. As demais depêndencias Python estão presentes no requirements.txt disponíveis online no repositório oficial deste projeto que se encontra em \href{https://github.com/antunesvitor/SimuladorDeConducao}{https://github.com/antunesvitor/SimuladorDeConducao}

Tanto a Unity quanto o Python estão presentes em para Windows e Linux, ambos os sistemas operacionais foram usados para o desenvolvimento deste projeto, porém grande parte dos testes foram conduzidos no \textbf{Arch Linux} utilizando-se do kernel \textbf{6.4.10-arch1-1}.

%Inserir aqui a versão do windows

No repositório mencinado acima encontra-se também instruções de instalação tanto no \textbf{Windows} quanto no \textbf{Arch Linux}.

% Criar um README.md no repositório com as instruções
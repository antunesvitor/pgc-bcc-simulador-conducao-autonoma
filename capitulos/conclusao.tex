\chapter*{Conclusões e Trabalhos Futuros}\label{cap:conclusao}
\addcontentsline{toc}{chapter}{Conclusão e Trabalhos Futuros}

Ao longo deste projeto, foi perceptível a dificuldade de se criar uma IA para veículo autônomo apesar de a duração dos treinos ter sido rápida (o desafio geral com PPO e BC convergindo com cerca de 90 minutos), vale lembrar que este simulador trata-se de um modelo excessivamente simples. Por outro lado, o agente foi bastante eficiente em realizar sua tarefa durante os testes após os treinos. Ficou notada a vantagem do Aprendizado por Reforço com PPO em relação ao Aprendizado por Imitação com BC e GAIL, principalmente este último que se mostrou bastante instável. Este projeto conseguiu cumprir o objetivo de provar ser possível criar um ambiente de treinamento para um agente capaz de conduzir um automóvel.

\section*{Trabalhos Futuros}

Podemos dividir os estudos futuros em três frentes: melhorias no modelo e no agente e exploração de novos algoritmos.

\subsection*{Melhorias no modelo e simulação}
Algumas das melhorias mais óbvias que podem ser feitas são a adição de outros elementos estáticos no cenário que estão ausentes nesta versão, por exemplo, semáforos, postes, obstruções na pista, prédios, etc. 

Como foi dito, este modelo atualmente é muito simples, o desafio de se criar um agente completamente capaz de conduzir um veículo em ruas públicas é um caminho longo e envolve muitos dilemas éticos. Felizmente, a Unity3D fornece uma gama de ferramentas que facilita a criação de cenários hipotéticos para testar o agente. Por exemplo, treinar um cenário multi-agente, onde poderíamos ter dentro da mesma simulação mais de um agente circulando nas ruas, além de outros veículos não inteligentes dirigidos por humanos. A simulação com pedestres talvez seja a mais importante de se simular, pois, um dos maiores temores que se tem com a tecnologia de veículos autônomos é como ela reagiria em um acidente iminente. Além do fato de que um condutor que esteja sempre atento e não cometa erros típicos humanos é um dos grandes objetivos da criação de carros autônomos.

\subsection*{Melhorias no agente}
Aqui foi testado um agente que consiga percorrer trajetos extremamente curtos, o agente tinha um número máximo de 800 \textit{steps} por rota para se chegar ao destino, o que dura em torno de 15 segundos. Este limite foi colocado para evitar que o veículo ficasse imóvel e o algoritmo preso em um mínimo local, porém, seria interessante aumentar o limite de \textit{steps} com o tamanho de trajetos, trajetos longos que demorassem pelo menos alguns minutos poderiam gerar um novo caso de estudos.

Outro problema com o agente foi a falta de uso de processamento de imagem, ele dispõe apenas de sensores de distância que conseguem detectar alguns elementos como \textit{checkpoint} e calçada. Com o uso de imagem o agente teria além de mais informação ele a disposição as mesmas observações que um humano possui enquanto dirige.

\subsection*{Exploração de novos algoritmos}
Como foi visto nos resultados, GAIL teve treinos instáveis não convergindo algumas vezes além de ter desempenhado pior nos testes, por outro lado, isso não descarta totalmente o algoritmo. Estes problemas podem ter sido causado por sensibilidade nos hiperparâmetros e uma exploração maior deles pode resultar em um treino mais estável e testes mais bem sucedidos. Como foi mencionado na fundamentação teórica, o pacote ML-Agents da Unity possui SAC como um dos algoritmos disponíveis, sendo que este foi criado justamente para ser menos sensível aos hiperparâmetros comparado a outros métodos, uma combinação SAC+GAIL pode gerar resultados melhores dos que foram vistos aqui.